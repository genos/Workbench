\documentclass[11pt]{article}
\usepackage{amsmath, amssymb, color, fullpage, listings, upquote, xfrac}
\definecolor{dkgreen}{rgb}{0, 0.6, 0}
\definecolor{mauve}{rgb}{0.58, 0, 0.82}
\lstset{backgroundcolor=\color{white},
        basicstyle=\footnotesize\ttfamily,
        breaklines=false,
        columns=[l]flexible,
        commentstyle=\color{dkgreen},
        keywordstyle=\color{blue},
        language=Python,
        numbers=none,
        showspaces=false,
        showstringspaces=false,
        stringstyle=\color{mauve}
}

\begin{document}

\section*{What are Continued Fractions?}

I was reading a great book recently, and it got me thinking about continued
fractions.\footnote{Yeah, yeah, sometimes I read really nerdy books.}
A continued fraction is something that looks like
\[
    n_0 +
        \cfrac{n_1}
            {d_1 + \cfrac{n_2}
                {d_2 + \cfrac{n_3}
                    {d_3 + \cfrac{n_4}
                        {d_4 + \cfrac{n_5}
                            {\ddots}
                        }
                    }
                }
            }
\]
where the diagonal dots mean the process goes on forever like a bottomless
well.
At this point we have no idea what kind of criteria the $\left(n_i, d_i\right)$
pairs need to satisfy for this to come out to an actual number, but don't worry
about that.
We won't go into lots of detail about continued fractions; we'll just play with
a very special one.

\section*{A Special Continued Fraction}

Let's look at the following:
\[
    1 +
        \cfrac{1}
            {1 + \cfrac{1}
                {1 + \cfrac{1}
                    {1 + \cfrac{1}
                        {1 + \cfrac{1}
                            {\ddots}
                        }
                    }
                }
            }
\]
Since we have no idea what it is, we might as well give it a name; call it $x$.
At the end of this note you'll find a \texttt{Python} program that will
approximate $x$ by going down the bottomless only a finite number of steps; you
might be able to run it on your own computer, or there are a number of
\texttt{Python} interpreters online---a few are listed in a comment above the
program's main code.
Don't run it yet, though!
You'll ruin the surprise.

\section*{Now what?}

I'm not an expert in continued fractions, so (perhaps like you) I don't know
the first thing about dealing with them in general.
I found something that will help us with this particular one, but more on that
later.

I had an incredible math teacher in High school\footnote{Ask Mr. Bill about
Paul Machemer sometime.} who had a great attitude about tackling hard problems:
\begin{quotation}
    \textit{If you don't know what to do, do something.}
\end{quotation}
The wisdom of continually trying things until something works can't be
overstated; you're bound to get somewhere eventually.
Even if you don't, you will end up knowing more about the problem than you did
before, and that's just as important.

\section*{OK $x$, just what are you?}
Let's look at $x$ again:
\[
    x = 1 +
        \cfrac{1}
            {1 + \cfrac{1}
                {1 + \cfrac{1}
                    {1 + \cfrac{1}
                        {1 + \cfrac{1}
                            {\ddots}
                        }
                    }
                }
            }
\]
Remember that the dots mean that the fractions continue on forever and ever.
If you look closely, you might notice something strange: the stuff under the
first fraction bar is just $x$ all over again!
This means
\[
    x = 1 + \frac{1}{x}
\]
with which I feel much more comfortable.
Multiplying through by $x$ to get rid of the denominators, we have
\begin{equation*}
    \begin{aligned}
        &x^2 = x + 1\\
        \implies &x^2 - x - 1 = 0
    \end{aligned}
\end{equation*}
Using the handy-dandy quadratic formula\footnote{Did you know that it fits to
\textit{Jingle Bells}? True story.}, we have
\begin{equation*}
    \begin{aligned}
        x &= \frac{1 \pm \sqrt{1^2 - 4 \cdot 1 \cdot 1}}{2}\\
          &= \frac{1 + \sqrt{5}}{2} \textrm{ or } \frac{1 - \sqrt{5}}{2}
    \end{aligned}
\end{equation*}

Although we have two possible answers for $x$, only one makes sense.
Note that the first one is positive, while the second one is negative (check it
with your calculator, don't just take my word for it); there was no subtraction
or negative numbers or any of that nonsense in our original continued fraction,
so the negative value doesn't make sense as an answer.\footnote{There's room
for a whole lot more detail here, but I'll skip it. If you're really
interested, I have a short book (covered with my own notes and scribbles,
sorry) that explores this much more in depth.}
That means that we have
\[
    x = \frac{1 + \sqrt{5}}{2}
\]
also known as the $\varphi$, golden ratio!
That thing pops up everywhere!\footnote{If you don't believe me, check out
``Donald Duck in Mathmagic Land,'' arguably the greatest movie of all time. I
wouldn't be surprised if there was a copy up on YouTube, not that I endorse
copyright infringement of course.}


\textbf{Happy Mathing!}\textit{---Mr. Bill's son Dr. E}
\pagebreak
\lstinputlisting{x.py}
\end{document}
